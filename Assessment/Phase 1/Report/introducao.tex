O presente relatório desenvolve-se no âmbito da Unidade Curricular \textit{System Deployment and Benchmarking}, tendo como objectivo expor o trabalho realizado nesta fase (seleção, caracterização e análise de uma aplicação distribuída), perspetivando uma estrutura fundamentada para a fase seguinte (automatização do processo de \textit{deployment} e de \textit{benchmarking}).
\par
Optou-se por seleccionar a aplicação \textit{Mattermost}, que é uma plataforma de mensagens instantâneas segura \textit{open-source}, direcionada ao trabalho de equipa, oferecendo a todos os utilizadores facilidade para conversar em \textit{chats} (privados ou públicos). O ponto forte desta aplicação é a possibilidade de hospedagem e centralização da plataforma de comunicação, de forma privada, numa \textit{NAS} local, permitindo o aumento do controlo, da disponibilidade, da confidencialidade e um alto potencial de armazenamento.
Os utilizadores podem interagir com o sistema através das aplicações \textit{desktop},  \textit{mobile} e da \textit{webapp}.
\par
Como era requisito do enunciado, este relatório comporta a descrição da arquitectura e componentes da aplicação, padrões de distribuição usados e formas de comunicação, a descrição dos pontos de configuração e a identificação de operações críticas, que representam possíveis \textit{bottlenecks} de desempenho. 
\par
Com a realização deste trabalho prático, o grupo objectiva uma melhor compreensão do processo de \textit{benchmarking} e de \textit{deployment}.
\par